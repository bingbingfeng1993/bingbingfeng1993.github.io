\chapter{总结与展望}

本文采用复杂网络的方法对几种典型的数字混沌系统的动力学性质进行了研究,主要研究成果如下:
\begin{itemize}

  \item

  利用整数量化函数准确推导出了定点运算域上一维混沌系统的状态映射网络(SMN)$F_e$与$F_{e+1}$对应节点间的强相关性;
  以Logistic映射、Tent映射为例,对定点运算域上一维混沌系统的SMN\ $F_e$与SMN\ $F_{e+1}$对应节点间的关系进行了具体分析。

  \item

  理论上证明了定点运算域上Logistic映射状态网络的无标度属性;
  厘清了浮点运算域上系统量化误差对Logistic映射运算结果的影响,分析发现,在浮点运算中,运算顺序和数值范围会对运算结果产生影响;
  根据有限域上混沌系统的状态映射网络可对混沌伪随机数发生器进行分类,并可从本质结构上查找常规检测工具很难发现的随机性缺陷。

  \item

  严格证明了数字计算机上Tent映射迭代值趋于零所需迭代次数$N_r$;
  准确推导出了定点运算域上Tent映射状态网络的入度分布,Tent映射状态网络的节点仅有3种可能入度取值;
  一定程度上厘清了定点和浮点运算域上混沌系统状态映射网络之间的关系,即浮点运算域上混沌系统的状态映射网络可以看作定点运算域上其对应状态映射网络的子网络重构。

  \item

  对有限域上二维离散Cat映射周期为$T$的不同Cat映射的数量$N_T$与周期$T$之间的关系进行了分析讨论;
  采用降维的方法揭示了定点运算域上二维离散Cat映射SMN\ $F_e$与SMN\ $F_{e+1}$之间的关系;
  通过有限域$\mathbb{Z}_{2^e}$上的随机实验发现二维离散Cat映射中环的分布呈指数为1的幂律分布。

\end{itemize}

在本论文的准备过程中,作者受限于水平和时间,有些问题没有得到妥善解决,这里简列如下:
(1) 定点和浮点运算域上混沌系统状态映射网络之间的具体关系有待进一步深入探讨;
(2) 二维离散Cat映射的环分布有待进一步理论证明。

